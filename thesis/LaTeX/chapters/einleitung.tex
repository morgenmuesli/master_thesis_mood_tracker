\chapter{Motivation}
% Was sind moodtracker (defintion, features etc), warum sind diese sinnvoll (Behandlung), Warum
The world health organization (WHO) showed that 20 percent of children and adolescents suffer from mental health conditions. Suicide is the second most death reason among 19-29 years olds\cite{who}.
With shortages in the number of therapists, there is a need for computer-assisted psychotherapy (CAT). 
Self reports and retrospective of habits and feelings are a fundamental concept in improving mental health.
For therapist it's the only chance to get an closer insight in the patients behavior during real live scenarios which helps them to find good strategies and treatments.
The problem with descriptions of events, feelings or behavior that lie in the past is that they usually do not correspond to reality and vary greatly. 
Memories are changed by external circumstances. Thus, negative experiences are perceived more strongly than positive ones\cite{bradburn1987answering}.
Ecological Momentary Assessment (EMA) or diary studies address this issue\cite{shiffman2008ecological}.
In this type of reports, the patient describe their habits on a daily basis, which results in a less biased retrospective because the events are still present.
A additional benefit, Daily journal writing helps patients, to train their own mindfulness and align their own focus on the progress they make. For example Hirage et al. showed that writing a diary can help people after a surgery to set and achieve own goals in their treatment\cite{hiraga2019effects}.
Even for healthy people, journaling can help reduce anxiety and stress and reduce the risk of mental illness.
Designing a text offers a cognitive difficulty, making it more difficult to access and integrate into daily routines.
Mood Tracker apps address this issue by providing a more easy way to track their emotions on their smartphone. 
Online Therapeutic tools like "Moodscope" showed scientific proven improvements of the users mental health\cite{drake2013assessing}. Therefore, mood trackers are serious category in CAT and 14.2 \% of all mobile health applications are mood tracker.
Although the increasing number of mood trackers is a good thing, the quality of these apps varies greatly. 
Scientific reviews of these criticize that many apps are developed without the instruction of psychological professionals and are more in line with the opinions and wishes of users\cite{caldeira_mobile_nodate}\cite{schueller2021understanding}.
But what is the opinion of the users? 
What exactly do users expect from this type of application and how is it implemented? 
While earlier attempt use randomly selected user reviews\cite{caldeira_mobile_nodate}, user interviews\cite{schueller2021understanding} or the mobile Mobile Application Rating Scale (MARS)\cite{myers2020evaluating} we want to use an natural language approach to cover those questions. 
We want to know which issues does user have with those applications and if those issues are in common with earlier researches. Also we want to cover, if academic designs for record mood behavior are in common with user practice. 
Is it more important to track the emotions precisely or is an easy and more accessible representation such as Emojis good enough for user satisfactions?






%Warum braucht es eine neue Studie?
%Main problem with current studies is that they only used randomized samples of the user reviews for predicting their need. Also
% interviews where held which suffer from bad generalizable therefor we are using an NLP based method for analyzing them
% easier to reproduce. 