\chapter{Research Question}
\section{App features}
Those questions are resulting in the feature analysis of the app itself.

\textbf{RQ\_AF01: What kind of representation is used to measure Emotions?}

As Caldeira showed in their research those apps offers different ways to measure emotions\cite{caldeira_mobile_nodate}.
We want to collect the different kinds of representations and look what trend is used to measure them.


\textbf{RQ\_AF02: What data other than emotions can be collected?}

Most of the apps provide additional tracking options, for example sleep conditions.
We want to know what categories are being tracked by the apps beside the emotions.

\textbf{RQ\_AF03: What type of methods are used to motivate users to report their feelings on a daily basis?}

We want to know if there are any patterns to motivate users to use the app on a daily basis.
Are there any reward systems or do they rely more on notifications? 

\textbf{RQ\_AF04: What kind of visualization is used to see review the tracked moods?}
Also the reports are showed differently, you can display your emotions inside a calendar or with a line-chart.
Another option is to see some distribution of your emotions over the last days.
We want to measure, which representations are used within the applications?

\section{User Reviews}
Those questions are resulting on the analyse of user reviews.

\textbf{RQ\_UR01: What are the main reasons in general to like those apps?}

We want to find the main reasons users review for giving those apps a good rating (>4 stars).
Are there any patterns for which does all those apps have in common?

\textbf{RQ\_UR02: What are the main reason in general to dislike those apps?}

What are the reasons in general for dislike those apps? 
Based on review keywords, we want to analyze which topics are the most present in user reviews.

\textbf{RQ\_UR03: How important are different representations of emotions?}

We want to know if other representations than likertscales between good and bad are preferred by users.
While emotions are multidimensional, a limitation to one dimension could be an argument against those applications.
It also categorized the emotions in "good emotions" and "bad emotions" which can be in contrast to the acceptance of negative valences emotions.
In academic research, multidimensional metrics such as the Self-Assessment Manikin\cite{bradley1994measuring} are the preferred choice, but what is the opinion in non-academic self reports?
Based on user reviews, we want to determine how important such metrics are in the functioning of the app.


\textbf{RQ\_UR04: How important are goals for users?}

Caldeira et al. criticized in their review that few apps only track the state and can't be used for setting goals\cite{caldeira_mobile_nodate}.
Those self defined aims can help to work on own patterns and was one of the examine user wishes to improve those apps.
On the one hand, we want to examine whether this trend has changed and on the other see how important this feature actually is for users.  

\textbf{RQ\_UR05: How important is privacy for users of these apps?}

Because of the personal data, privacy concerns was another user issue Caldeira found.
However different studies showed that there is a discrepancy between privacy concerns and the actual behavior\cite{BARTH20171038}.
We want to know if online availability of data is more important than secure the data only on local storage.

\textbf{Has the interest in Mood Tracker increased during the Corona crisis? (optional)}

Social Isolation and loneliness was one of the big challenges during the pandemic\cite{loneliness}.
We want to now if this has an effect in the interest in mood tracker as a type of social support.

