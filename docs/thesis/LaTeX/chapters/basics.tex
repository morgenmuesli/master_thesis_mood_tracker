\chapter{Fundamentals and related work}
\label{ch:basics}

\section{Introduction}

In this chapter we will present some basic concepts of cognitive behavior therapie, 
which is the scientific basis of mood trackers.
We will also present some related work in the field of mood trackers and their evaluation.
In the second part we will introduce the basic concepts of natural language processing and the used methods.

\section{Cognitive Behavior Therapy}

Cognitive behavior therapy (CBT) is a treatment, which shown to be effective to reduce symptoms of several mental disordes 
included depression and anxiety.
It is based on the idea that our thoughts, beliefs and behaviors are interconnected.
The aim is to find and change cognitive distortions and maladaptive behaviors.

% TODO:: UMSCHREIBEN!!!!
The core principes of cbt are
\begin{itemize}
    \item that the psychological problems can be based on unproductive (fault or unhealthy) ways of thinking.
    \item the behavior or learned patterns can be the problem of the psychological problems.
    \item A different coping mechanism can be learned to reduce the psychological problems.
\end{itemize}

The strategy of that is that the patient learned to recognize and reevalute the thought that are creating the problems.
As an example the thought "I am a failure" can be recognized and reevaluated to "I am not a failure, I just failed this time" 
when confronting with earlier results.
Another strategy is to develop a better perception about the own abilities in order to strengthen the self-esteem.
Unlike other therapies, CBT relies on the patient learning to be their own therapist 
and recognizing unhealthy behaviors or false perceptions themselves. 

\section{Mood trackers}



