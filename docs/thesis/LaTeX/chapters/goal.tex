%% goal.tex
%% ==============================
\chapter{Goal of this thesis}
\label{ch:goal}
%% ==============================
\section{Problem}
\label{sec:problem}
% earlier reviews don't cover all reviews
% mostly look for mental disorders exclusively

There is only a limited amount of studies which focused on mood tracking applications in general.
Many of the reviews are targeting mental illnesses such as bipolar disorders, depression or anxiety as a motivation to use those application \cite{nicholas2015mobile}\cite{Balaskas2022UnderstandingUP}.
Despite the fact that this can be an important motivation for some people, it is not the only one.
Caldeira et al. \cite{caldeira2017mobile} showed in their study that the most motivation for user was to learn own mood patterns. Also Schueller et al. \cite{schueller2021understanding} showed in their interview that also healthy people are using those apps for self awareness after a bad events in their life.
They used those apps as a way to reflect on their life and to improve their self esteem.
This indicates that the target group of those applications is not only people with mental disorders but also without.
Reviews which are targeting mental disorders can miss important aspects of those applications which are important for healthy people.
As we can see in the interview study from Schueller people with good mood wanted other features (eg. recommendations) than people which are currently depressed \cite{schueller2021understanding}. However such desired features can be annoying for users. Recommendations and notifications in those applications for example, can be motivating for some users\cite{widnall2020user} but can also annoying \cite{Balaskas2022UnderstandingUP}.
The major limitation of all studies is, that non of the study covering all user reviews. 
Caldeira \cite{caldeira2017mobile}, Balaskas et al. \cite{Balaskas2022UnderstandingUP} and Widnall \cite{widnall2020user} are selecting random reviews and read them by their self. 
Although this captures the context of the reviews very clearly, it means that a large proportion of user reviews are ignored. 
Added interview studies, such as those used by Balaskas \cite{Balaskas2022UnderstandingUP} and Schueller \cite{Balaskas2022UnderstandingUP}, improve internal validity but can also be criticized for their external validity.
We want to use modern Data Mining Approaches for getting a deep insight into the user responses about app features.

\section{Research objective}
\label{sec:research-objective}
The goal of this thesis is to get a good insight over what users expect from those applications.
We first want to get an overview over the different features of the different mood tracking apps.
Than we analyze which features are important for the users and what features aren't mention in the reviews.
The user target should be as general as possible and we want to cover all reviews instead of a random sample.
As second step we want to analyze, if therapeutic features which are important for experts are also wished by users.
\subsection{Research questions}
\label{sec:research-questions}
\begin{enumerate}
\item \textbf{Research Goal 1:}
We want to compile feature overview of mobile mood tracking
apps for the average population.
\item \textbf{Research Goal 2:} In addition to those features we investigate potential reasons
for their success or failure based on the user reviews
\end{enumerate}


\section{Expected results}
\label{sec:expected-results}
We expect that user are happy with less but reasonable features.
We think that the most of therapeutic features such as goals, 
predefined tasks or medication features aren't relevant for most people and they want 
a reduced application with only mood and behavior tracking.
We also think that an application with less features but good user interface have a better review score, than rich featured app which can be overwhelming.



