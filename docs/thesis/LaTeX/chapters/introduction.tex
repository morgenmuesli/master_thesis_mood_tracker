%% introduction.tex
%%
\chapter{Introduction}
\label{ch:introduction}
The world health organization (WHO) showed that 20 percent of children and adolescents suffer from mental health conditions. 
Suicide is the second most death reason among 19-29 years olds\footnote{\url{https://www.who.int/news-room/fact-sheets/detail/suicide} Accessed on 2021-11-01}.
With shortages in the number of therapists, there is a need for computer-assisted psychotherapy (CAT). 
Self reports and retrospective of habits and feelings are a fundamental concept in improving mental health.
The problem with descriptions of events, feelings or behavior that lie in the past is that they usually do not correspond to reality and vary greatly. 
Memories are changed by external circumstances.
Ecological Momentary Assessment (EMA) or diaries address this issue \cite{shiffman2008ecological}.
In this type of application, the patient describe their emotions and habits on a daily basis, which results in a less biased retrospective because the events are still present.
The benefit, Daily journal writing helps patients, to train their own mindfulness and align their own focus on the progress they make \cite{hiraga2019effects}.
Even for healthy people, journaling can help reduce anxiety and stress and reduce the risk of mental illness.
Designing a text offers a cognitive difficulty, making it more difficult to access and integrate into daily routines.
The goal of these applications is to give the user a simple and fast alternative to diaries and to get an overview of his emotional state by means of predefined answers.
Online Therapeutic tools like "Moodscope" showed scientific proven improvements of the users mental health \cite{drake2013assessing}. Therefore, mood trackers are serious category in CAT and 14.2 \% of all mobile health applications are mood tracker.
Although the increasing number of mood trackers is a good thing, the quality of these apps varies greatly. 
Scientific reviews of these criticize that many apps are developed without the instruction of psychological professionals and are more in line with the opinions and wishes of users \cite{caldeira2017mobile}\cite{schueller2021understanding}.
But what is the opinion of the users? 
What exactly do users expect from this type of application and how is it implemented? 
While earlier attempt use randomly selected user reviews \cite{caldeira2017mobile}, user interviews \cite{schueller2021understanding} or the mobile Mobile Application Rating Scale (MARS) \cite{myers2020evaluating} we want to use an natural language approach to cover those questions. 
We want to know which issues does user have with those applications and if those issues are in common with earlier researches. Also we want to cover, if academic designs for record mood behavior are in common with user practice. 
