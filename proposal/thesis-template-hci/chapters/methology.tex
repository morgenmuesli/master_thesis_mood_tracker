%% methology.tex
%% ==============================
\chapter{Methology}
\label{ch:methology}
%% ==============================
\section{Data collection}
\label{sec:data-collection}
This study is divided in two separated parts. The first part is a systematic review of the applications. We follow the guidelines of the reporting checklist of the Preferred Reporting Items for Systematic Reviews and Meta-Analysis (PRISMA\cite{moher2015preferred}).
We define mood trackers as:
"Applications for measuring and reporting mood by themselves on a daily basis".
We exclude applications which are:
\begin{itemize}
    \item main focus is not mood tracking (eg. \textit{Daily Diary: Journal with Lock} \footnote{\url{https://play.google.com/store/apps/details?id=com.daily.journal.diary.lock.mood.tracker.free&hl=en&gl=US} Accessed on 2021-11-01})
    \item doesn't collect mood behavior (eg. journalistic \footnote{\url{https://play.google.com/store/apps/details?id=com.journalisticapp.twa&hl=en&gl=US} Accessed on 2021-11-01})
    \item tracking others behavior such as parenting applications or relationship applications (eg. \textit{behavior diary} \footnote{\url{https://play.google.com/store/apps/details?id=in.co.skycap.behaviourtracker&hl=en&gl=US} Accessed on 2021-11-01})
    \item only available as a web app and not included in any applications store (eg. \textit{moodtracker.com}\footnote{\url{https://www.moodtracker.com/} Accessed on 2021-11-01})
    \item targeting a specific group of people (eg. \textit{Bipolar Mood Tracker}\footnote{\url{https://play.google.com/store/apps/details?id=com.bipolar_flutter&hl=en&gl=US} Accessed on 2021-11-01})
\end{itemize}
As data source we are using the google play store\footnote{\url{https://play.google.com/store/games?hl=en&gl=US} Accessed on 2021-11-01 } as well as the apple appstore because the two operation systems covers more than 99\% of the worldwide mobile operation system market share.
Our search queries are: ["mood tracker", "mood journal", "mood ema", "emotion tracker"].
For feature extraction we are using the app descriptions as raw data. We include only applications which are available at the time period of our study (November 2022).
The User Reviews are crawled with appbot\footnote{\url{https://appbot.co/} Accessed: 07.11.2022}.

\section{Analyzing User Reviews}
%% ==============================
\subsection{Preprocessing}
We group our data into positive, neutral and negative reviews using the star ratings.
As negative we define reviews with 1 or 2 stars, as neutral reviews with 3 stars and as positive reviews with 4 or 5 stars.
\label{sec:analyzing-user-reviews}
For analyzing the user reviews we are using two types of data mining.
In our first study we are using n-gram frequency to get a first impression on the most important features.
In our second study we are using topic modeling to get a better understanding of the context of the reviews.
\subsection{N-gram frequency}
\label{sec:n-gram-frequency}
Just apply the term frequency can lead to rather meaningless results.
The reason for this is, that some words are more used in the corpus itself and doesn't give any information about the content of the reviews.
Instead we using the tf-idf (term frequency-inverse document frequency) to get a better understanding of the content of the reviews.
It is a combination of the term frequency and the inverse document frequency.
The term frequency is the number of times a word appears in a document.
The inverse document frequency is the logarithm of the number of documents divided by the number of documents in which the word appears.

However using single words as features can lead to a lot of noise.
Since their are many terms which belongs together. For example "battery life" or "user interface".
To adress this issue we are using n-grams, which are sequences of n words.



\subsection{Topic modeling}
\section{Evaluate the results}

For analyzing our results we are using the following metrics:
