%% goal.tex
%% ==============================
\chapter{Goal of this thesis}
\label{ch:goal}
%% ==============================
\section{Problem}
\label{sec:problem}
% earlier reviews don't cover all reviews
% mostly look for mental disorders exclusively

Allthough such applications are reviewed by experts we explore some limitations according their methods.
Many of the reviews are targeting mental illnesses such as bipolar disorders, depression or anciety as a motivation to use those applications.
Despite the fact that this can be an important motivation for some people, it is not the only one.
Schueller et al. \cite{schueller2021UnderstandingPU} showed in their interview study that many healthy people are using those apps for self awereness after a bad events in their life.
They used those apps as a way to reflect on their life and to improve their self esteem.
This indicates that the target group of those applications is not only people with mental disorders but also without.
Reviews which are targeting mental disorders can miss important aspects of those applications which are important for healthy people.
Another limitation is that non of the reviews covering all user reviews.
Caldeira \cite{caldeira_mobile_nodate} and Balaskas et al. \cite{Balaskas2022UnderstandingUP} are selecting random reviews and read them by theirself.
Although this captures the context of the reviews very clearly, it means that a large proportion of user reviews are ignored.
Added interview studies, such as those used by Balaskas \cite{Balaskas2022UnderstandingUP} and Schueller \cite{Balaskas2022UnderstandingUP},
improve internal validity but can also be criticized for their external validity.
Modern approaches which includes data mining can cover much more reviews but suffer in terms of internal validity.
The goal of this thesis is to find a method which covers all user reviews and is able to capture the context of the reviews.

\section{Research objective}
\label{sec:research-objective}

In this study we want to validate the findings of earlier studies based on natural language processing (NLP) appraoches.
We want to explore if modern NLP approaches are usable for this analysis and
if the findins of earlier studies can be validate or disprove based on our results on all reviews.
\subsection{Research questions}
\label{sec:research-questions}
\begin{enumerate}
\item \textbf{RQ1:} Can modern NLP approaches be used to analyze user reviews in context of feature importance?
\item \textbf{RQ2:} Are the findings of those approaches can be used for validate or disprove the findings of earlier studies?
\end{enumerate}


\section{Expected results}
\label{sec:expected-results}
We expect that the general user oppinion differs from the app design oppinion of the experts. 
The expection is that users want to have an easy to use app with a good design instead of having a lot of therapeutic features.





